\documentclass[12pt,a4paper]{article}
\usepackage[utf8]{inputenc}
\usepackage[french]{babel}
\usepackage[T1]{fontenc}
\usepackage{geometry}
\usepackage{graphicx}
\usepackage{hyperref}
\usepackage{fancyhdr}
\usepackage{xcolor}
\usepackage{listings}
\usepackage{tocloft}
\usepackage{titlesec}

\geometry{margin=2.5cm}

% Configuration des couleurs
\definecolor{titlecolor}{RGB}{25,118,210}
\definecolor{sectioncolor}{RGB}{56,142,60}

% Configuration des en-têtes
\pagestyle{fancy}
\fancyhf{}
\fancyhead[L]{Projet Système d'Information}
\fancyhead[R]{Évaluation des Enseignants}
\fancyfoot[C]{\thepage}

% Configuration des titres de sections
\titleformat{\section}
  {\color{titlecolor}\normalfont\Large\bfseries}
  {\thesection}{1em}{}
\titleformat{\subsection}
  {\color{sectioncolor}\normalfont\large\bfseries}
  {\thesubsection}{1em}{}

% Configuration des liens
\hypersetup{
    colorlinks=true,
    linkcolor=blue,
    filecolor=magenta,      
    urlcolor=cyan,
    pdftitle={Rapport Projet SI},
}

\begin{document}

% Page de garde
\begin{titlepage}
    \centering
    \vspace*{2cm}
    
    {\Huge\bfseries Projet Système d'Information\par}
    \vspace{1.5cm}
    
    {\LARGE Système d'évaluation des enseignants\par}
    \vspace{0.5cm}
    {\large Analyse et Conception avec Architecture MVC\par}
    
    \vspace{3cm}
    
    {\Large\textbf{Étudiants :}\par}
    \vspace{0.5cm}
    {\large 
    \textbf{Alexis CHARPENTREAU}\par
    \textbf{Arthur MAGNETTE}\par
    }
    
    \vfill
    
    {\large Année académique 2025-2026\par}
    {\large \today\par}
    
\end{titlepage}

% Table des matières
\tableofcontents
\newpage

% Introduction
\section{Introduction}

\subsection{Contexte du projet}

Ce rapport présente l'analyse et la conception d'un système d'information dédié à l'évaluation des enseignants par les étudiants. Le système vise à garantir l'anonymat des évaluations tout en fournissant des données statistiques pertinentes aux enseignants et aux administrateurs.

\subsection{Objectifs}

Les principaux objectifs du système sont :
\begin{itemize}
    \item Permettre aux étudiants de remplir des évaluations anonymes
    \item Fournir aux enseignants un tableau de bord interactif avec leurs statistiques
    \item Garantir la confidentialité et l'intégrité des données
    \item Offrir une interface responsive (desktop et mobile)
\end{itemize}

\subsection{Méthodologie}

Le projet a été réalisé en suivant une approche itérative comprenant :
\begin{enumerate}
    \item Analyse des cas d'utilisation
    \item Conception MVC (Modèle-Vue-Contrôleur)
    \item Modélisation avec UML (diagrammes de séquence, classes, états-transitions)
    \item Prototypage des interfaces utilisateur
    \item Modélisation des processus métier avec BPMN
\end{enumerate}

\newpage

\section{Diagramme de cas d'utilisation}

Le diagramme de cas d'utilisation global identifie les trois acteurs principaux du système :

\subsection{Acteurs}

\begin{itemize}
    \item \textbf{Étudiant} : Remplit les évaluations de manière anonyme
    \item \textbf{Enseignant} : Consulte ses résultats d'évaluation via un tableau de bord
    \item \textbf{Administrateur} : Gère le système, configure les évaluations, et accède aux statistiques globales
\end{itemize}

\subsection{Cas d'utilisation identifiés}

\begin{enumerate}
    \item S'authentifier (tous les acteurs)
    \item Remplir une évaluation (Étudiant)
    \item Consulter l'historique des évaluations (Étudiant)
    \item Visualiser le tableau de bord (Enseignant)
    \item Consulter les statistiques détaillées (Enseignant)
    \item Comparer les performances (Administrateur)
    \item Gérer les formulaires d'évaluation (Administrateur)
    \item Générer des rapports (Administrateur)
\end{enumerate}

\begin{figure}[H]
    \centering
    \includegraphics[width=0.8\textwidth]{Diagramme_Cas_Util}
    \caption{Diagramme de Cas d'utilisation}
    \label{fig:cas_util}
\end{figure}


\newpage

\section{Analyse des cas d'utilisation avec approche MVC}

Deux cas d'utilisation ont été sélectionnés pour une analyse approfondie selon l'architecture MVC.

\subsection{UC1 : Remplir une évaluation}

\subsubsection{Description}

L'étudiant accède au système, sélectionne une évaluation en attente, remplit le formulaire de manière anonyme, et soumet son évaluation. Le système garantit l'anonymat et la possibilité de sauvegarder un brouillon.

\subsubsection{Architecture MVC}

\textbf{Vue (Présentation)} :
\begin{itemize}
    \item \texttt{VueListeEvaluations} : Affichage de la liste des évaluations
    \item \texttt{VueFormulaireEvaluation} : Interface de saisie du formulaire
    \item \texttt{VueConfirmation} : Écran de récapitulatif
    \item \texttt{VueSucces} : Confirmation de soumission
    \item \texttt{VueErreurValidation} : Affichage des erreurs
\end{itemize}

\textbf{Contrôleur (Logique métier)} :
\begin{itemize}
    \item \texttt{EvaluationController} : Gestion du cycle de vie de l'évaluation
    \item Opérations : \texttt{afficherListe()}, \texttt{validerFormulaire()}, \texttt{soumettre()}
\end{itemize}

\textbf{Modèle (Données)} :
\begin{itemize}
    \item Services : \texttt{AnonymatService}, \texttt{ValidationService}, \texttt{NotificationService}
    \item Entités : \texttt{Evaluation}, \texttt{Formulaire}, \texttt{Question}, \texttt{Reponse}
\end{itemize}

\subsubsection{Scénarios modélisés}

Le diagramme de séquence couvre :
\begin{enumerate}
    \item Scénario idéal : Soumission réussie
    \item Alternative 1 : Erreurs de validation
    \item Alternative 2 : Sauvegarde en brouillon
    \item Alternative 3 : Erreurs système
\end{enumerate}

\begin{figure}[H]
    \centering
    \includegraphics[width=0.9\textwidth]{Diagramme_Sequence_UC1_Remplir_Evaluation}
    \caption{Diagramme de séquence - UC1 Remplir une évaluation}
    \label{fig:seq_uc1}
\end{figure}

\textit{Fichiers associés : \texttt{Diagramme\_Sequence\_UC1\_Remplir\_Evaluation.puml}}

\newpage

\subsection{UC6 : Visualiser le tableau de bord interactif}

\subsubsection{Description}

L'enseignant se connecte et accède à un tableau de bord présentant ses indicateurs de performance (KPI), l'évolution de ses notes, les commentaires anonymes des étudiants, et des statistiques détaillées par cours.

\subsubsection{Architecture MVC}

\textbf{Vue (Présentation)} :
\begin{itemize}
    \item \texttt{VueTableauBord} : Dashboard principal avec KPI
    \item \texttt{VueCommentaires} : Liste des commentaires anonymes
    \item \texttt{VueStatistiques} : Analyses détaillées par cours
    \item \texttt{VueFiltres} : Interface de filtrage
\end{itemize}

\textbf{Contrôleur (Logique métier)} :
\begin{itemize}
    \item \texttt{TableauBordController} : Orchestration de l'affichage
    \item Opérations : \texttt{chargerDashboard()}, \texttt{appliquerFiltres()}, \texttt{exporterPDF()}
\end{itemize}

\textbf{Modèle (Données)} :
\begin{itemize}
    \item Services : \texttt{StatistiquesService}, \texttt{EvolutionService}, \texttt{CommentairesService}
    \item Entités : \texttt{Enseignant}, \texttt{Cours}, \texttt{Statistique}, \texttt{KPI}
\end{itemize}

\subsubsection{Scénarios modélisés}

Le diagramme de séquence couvre :
\begin{enumerate}
    \item Scénario idéal : Chargement du tableau de bord
    \item Alternative 1 : Aucune donnée disponible (nouvel enseignant)
    \item Alternative 2 : Version mobile
    \item Alternative 3 : Application de filtres
    \item Alternative 4 : Erreurs de chargement
\end{enumerate}

\begin{figure}[H]
    \centering
    \includegraphics[width=0.9\textwidth]{Diagramme_Sequence_UC6_Tableau_Bord}
    \caption{Diagramme de séquence - UC6 Consulter le tableau de bord}
    \label{fig:seq_uc6}
\end{figure}

\textit{Fichiers associés : \texttt{Diagramme\_Sequence\_UC6\_Tableau\_Bord.puml}}

\newpage

\section{Diagramme de classes MVC}

Le diagramme de classes présente l'architecture complète du système selon le pattern MVC.

\subsection{Organisation par packages}

\textbf{Package Vue} :
\begin{itemize}
    \item 11 classes d'interface (UC1 et UC6)
    \item Responsabilité : Présentation et interaction utilisateur
\end{itemize}

\textbf{Package Contrôleur} :
\begin{itemize}
    \item \texttt{EvaluationController}
    \item \texttt{TableauBordController}
    \item Responsabilité : Logique de coordination
\end{itemize}

\textbf{Package Modèle} :
\begin{itemize}
    \item 5 services métier
    \item 9 entités du domaine
    \item Responsabilité : Logique métier et persistance
\end{itemize}

\subsection{Relations principales}

\begin{itemize}
    \item \textbf{Dépendance} : Vue $\rightarrow$ Contrôleur $\rightarrow$ Modèle
    \item \textbf{Association} : Entités liées (Étudiant-Évaluation, Enseignant-Cours)
    \item \textbf{Agrégation} : Formulaire contient Questions
    \item \textbf{Composition} : Évaluation contient Réponses
\end{itemize}

\begin{figure}[H]
    \centering
    \includegraphics[width=1\textwidth]{Diagramme_Classe_MVC}
    \caption{Diagramme de classes - Architecture MVC}
    \label{fig:classe_mvc}
\end{figure}

\textit{Fichier associé : \texttt{Diagramme\_Classe\_MVC.puml}}

\newpage

\section{Prototypes d'interfaces (IHM)}

Des maquettes interactives ont été créées avec PlantUML Salt pour les deux cas d'utilisation.

\subsection{UC1 : Interfaces d'évaluation}

\textbf{7 écrans développés :}
\begin{enumerate}
    \item \textbf{Liste des évaluations (Desktop)} : Vue d'ensemble avec filtres
    \item \textbf{Formulaire d'évaluation (Desktop)} : Saisie complète avec sections
    \item \textbf{Confirmation} : Récapitulatif avant soumission
    \item \textbf{Succès} : Message de confirmation
    \item \textbf{Liste des évaluations (Mobile)} : Version responsive
    \item \textbf{Formulaire (Mobile)} : Adaptation mobile
    \item \textbf{Erreur de validation} : Affichage des erreurs
\end{enumerate}

\begin{figure}[H]
    \centering
    \includegraphics[width=0.95\textwidth]{IHM_UC1_Remplir_Evaluation}
    \caption{Prototypes d'interfaces - UC1 Remplir une évaluation (7 écrans)}
    \label{fig:ihm_uc1}
\end{figure}

\subsection{UC6 : Interfaces du tableau de bord}

\textbf{8 écrans développés :}
\begin{enumerate}
    \item \textbf{Tableau de bord principal} : KPI, graphiques d'évolution, commentaires récents
    \item \textbf{Tableau de bord avec filtres} : Application de filtres par période/cours
    \item \textbf{Commentaires détaillés} : Liste paginée avec recherche
    \item \textbf{Statistiques détaillées} : Analyse par cours et par question
    \item \textbf{Comparaison} : Positionnement vs moyenne départementale (Admin)
    \item \textbf{Version mobile} : Dashboard adapté mobile
    \item \textbf{Aucune donnée} : État vide pour nouvel enseignant
    \item \textbf{Erreur de chargement} : Gestion des erreurs
\end{enumerate}

\begin{figure}[H]
    \centering
    \includegraphics[width=0.95\textwidth]{IHM_UC6_Tableau_Bord}
    \caption{Prototypes d'interfaces - UC6 Tableau de bord (8 écrans)}
    \label{fig:ihm_uc6}
\end{figure}

\subsection{Principes de conception}

\begin{itemize}
    \item \textbf{Responsive design} : Adaptation desktop/mobile
    \item \textbf{Progressive disclosure} : Information hiérarchisée
    \item \textbf{Feedback utilisateur} : États de validation clairs
    \item \textbf{Accessibilité} : Contraste et lisibilité
\end{itemize}

\textit{Fichiers associés : \texttt{IHM\_UC1\_Remplir\_Evaluation.puml}, \texttt{IHM\_UC6\_Tableau\_Bord.puml}}

\newpage

\section{Diagramme de navigation}

Le diagramme d'états-transitions modélise la navigation entre les différentes interfaces du système.

\subsection{Navigation UC1 (Étudiants)}

\textbf{Flux principal :}
\begin{itemize}
    \item Authentification $\rightarrow$ Liste évaluations
    \item Commencer évaluation / Reprendre brouillon $\rightarrow$ Formulaire
    \item Formulaire $\rightarrow$ Validation $\rightarrow$ Confirmation $\rightarrow$ Soumission
    \item Branche alternative : Sauvegarde brouillon $\rightarrow$ Retour liste
    \item Branche erreur : Validation échouée $\rightarrow$ Correction
\end{itemize}

\subsection{Navigation UC6 (Enseignants)}

\textbf{Flux principal :}
\begin{itemize}
    \item Authentification $\rightarrow$ Tableau de bord principal
    \item Tableau de bord $\leftrightarrow$ Filtres actifs
    \item Tableau de bord $\rightarrow$ Commentaires détaillés
    \item Tableau de bord $\rightarrow$ Statistiques détaillées
    \item Tous les états $\rightarrow$ Retour au tableau de bord
\end{itemize}

\subsection{États terminaux}

\begin{itemize}
    \item Déconnexion depuis n'importe quel état
    \item État final après soumission d'évaluation
\end{itemize}

\begin{figure}[H]
    \centering
    \includegraphics[width=0.85\textwidth]{Navigation_Globale}
    \caption{Diagramme de navigation - États et transitions}
    \label{fig:navigation}
\end{figure}

\textit{Fichier associé : \texttt{Navigation\_Globale.puml}}

\newpage

\section{Modélisation BPMN}

Le diagramme BPMN (Business Process Model and Notation) modélise le processus complet d'évaluation avec deux acteurs principaux.

\subsection{Pool 1 : Étudiants}

\textbf{Processus :}
\begin{enumerate}
    \item Recevoir notification d'évaluation par email
    \item Se connecter au système
    \item Consulter liste des évaluations
    \item Décision : Reprendre brouillon ou nouvelle évaluation
    \item Boucle de remplissage :
    \begin{itemize}
        \item Remplir formulaire (questions, notes, commentaires)
        \item Sauvegarde automatique
        \item Validation
        \item Si erreurs : correction
    \end{itemize}
    \item Afficher récapitulatif
    \item Décision : Confirmer ou modifier
    \item Soumettre évaluation définitive
    \item Recevoir confirmation par email
\end{enumerate}

\subsection{Pool 2 : Université}

\textbf{Processus :}
\begin{enumerate}
    \item Configurer période d'évaluation
    \item Générer formulaires par cours
    \item Envoyer notifications aux étudiants
    \item Processus parallèle :
    \begin{itemize}
        \item Recevoir évaluations soumises
        \item Contrôler intégrité des données
        \item Anonymiser (suppression ID, hash temporel, chiffrement)
        \item Stocker dans base anonymisée
    \end{itemize}
    \item À la fin de période :
    \begin{itemize}
        \item Agréger toutes les évaluations
        \item Calculer statistiques (moyennes, écarts-types, distributions)
        \item Analyser sentiments des commentaires (NLP)
        \item Fork parallèle :
        \begin{itemize}
            \item Générer rapport par enseignant
            \item Publier résultats sur tableau de bord
        \end{itemize}
        \item Notifier enseignants par email
        \item Archiver données pour audit
    \end{itemize}
\end{enumerate}

\subsection{Points d'interaction}

\begin{itemize}
    \item \textbf{Notification} : Université $\rightarrow$ Étudiant
    \item \textbf{Soumission} : Étudiant $\rightarrow$ Université
    \item \textbf{Confirmation} : Université $\rightarrow$ Étudiant
    \item \textbf{Résultats} : Université $\rightarrow$ Enseignant
\end{itemize}

\begin{figure}[H]
    \centering
    \includegraphics[width=1\textwidth]{BPMN_Processus_Evaluation}
    \caption{Diagramme BPMN - Processus complet d'évaluation avec 2 pools}
    \label{fig:bpmn}
\end{figure}

\textit{Fichier associé : \texttt{BPMN\_Processus\_Evaluation.puml}}

\newpage

\section{Aspects techniques}

\subsection{Technologies utilisées}

\begin{itemize}
    \item \textbf{UML / PlantUML} : Modélisation des diagrammes
    \item \textbf{PlantUML Salt} : Prototypage des interfaces
    \item \textbf{Architecture MVC} : Séparation des responsabilités
    \item \textbf{BPMN 2.0} : Modélisation des processus métier
\end{itemize}

\subsection{Garanties du système}

\begin{itemize}
    \item \textbf{Anonymat} : Suppression des identifiants, hash temporel, chiffrement
    \item \textbf{Intégrité} : Validation des données, transactions ACID
    \item \textbf{Disponibilité} : Sauvegarde automatique, gestion des erreurs
    \item \textbf{Confidentialité} : Accès restreint selon les rôles
\end{itemize}

\subsection{Scalabilité}

Le système est conçu pour supporter :
\begin{itemize}
    \item Plusieurs centaines d'étudiants par cours
    \item Évaluations simultanées en période de pic
    \item Archivage sur plusieurs années académiques
    \item Analyses statistiques complexes
\end{itemize}

\newpage

\section{Conclusion}

\subsection{Synthèse du travail réalisé}

Ce projet a permis de concevoir un système d'information complet pour l'évaluation des enseignants, en suivant une méthodologie rigoureuse et en utilisant les standards de modélisation UML et BPMN.

Les livrables produits comprennent :
\begin{enumerate}
    \item Diagramme de cas d'utilisation global
    \item Analyse détaillée de 2 UC avec architecture MVC
    \item 2 diagrammes de séquence complets (scénarios idéaux et alternatives)
    \item 1 diagramme de classes MVC (30+ classes)
    \item 15 prototypes d'interfaces (desktop et mobile)
    \item 1 diagramme de navigation (états-transitions)
    \item 1 diagramme BPMN (2 pools)
\end{enumerate}

\subsection{Compétences mobilisées}

\begin{itemize}
    \item \textbf{Analyse fonctionnelle} : Identification des besoins et des acteurs
    \item \textbf{Conception orientée objet} : Architecture MVC, séparation des responsabilités
    \item \textbf{Modélisation UML} : Diagrammes de séquence, classes, états
    \item \textbf{Prototypage IHM} : Conception d'interfaces utilisateur
    \item \textbf{Modélisation de processus} : BPMN avec swimlanes
\end{itemize}

\subsection{Perspectives d'évolution}

Le système pourrait être enrichi avec :
\begin{itemize}
    \item Intelligence artificielle pour l'analyse automatique des commentaires
    \item Système de recommandations pour l'amélioration pédagogique
    \item Intégration avec d'autres systèmes universitaires (LMS, RH)
    \item Tableau de bord administratif avec analytics avancés
    \item Module de questionnaires personnalisables
\end{itemize}

\newpage

\section*{Annexes}

\subsection*{Liste des fichiers livrés}

\begin{enumerate}
    \item \texttt{Diagramme\_Cas\_Util.puml} - Diagramme de cas d'utilisation
    \item \texttt{Diagramme\_Sequence\_UC1\_Remplir\_Evaluation.puml} - Séquence UC1
    \item \texttt{Diagramme\_Sequence\_UC6\_Tableau\_Bord.puml} - Séquence UC6
    \item \texttt{Diagramme\_Classe\_MVC.puml} - Diagramme de classes
    \item \texttt{IHM\_UC1\_Remplir\_Evaluation.puml} - Prototypes IHM UC1
    \item \texttt{IHM\_UC6\_Tableau\_Bord.puml} - Prototypes IHM UC6
    \item \texttt{Navigation\_Globale.puml} - Diagramme de navigation
    \item \texttt{BPMN\_Processus\_Evaluation.puml} - Diagramme BPMN
    \item \texttt{Analyse\_UC\_MVC.md} - Documentation de l'analyse MVC
    \item \texttt{Documentation\_IHM.md} - Documentation des interfaces
\end{enumerate}

\subsection*{Références}

\begin{itemize}
    \item UML 2.5 Specification - Object Management Group (OMG)
    \item BPMN 2.0 Specification - Object Management Group (OMG)
    \item PlantUML Documentation - \url{https://plantuml.com}
    \item Design Patterns - Gang of Four (GoF)
    \item Model-View-Controller Pattern - Trygve Reenskaug
\end{itemize}

\end{document}
